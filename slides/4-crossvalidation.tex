% * Week 3. (Aug 10) Regression for prediction. Ch3 (Rob)
%   - Lecture 5: Review of linear regression, matrix formulation
%   - Lab 3:
%   - Lecture 6: Subset selection, LOOCV 


\documentclass[14pt]{beamer}
\usepackage{pgf,tikz,pgfpages,amsmath,bm,fancyvrb,animate}
\usepackage{graphicx,bera,booktabs}
\usepackage[australian]{babel}
\usepackage[utf8]{inputenc}

\usetheme{Monash}
\def\biz{\begin{itemize}[<+-| alert@+>]}
\def\eiz{\end{itemize}}
\def\ben{\begin{enumerate}[<+-| alert@+>]}
\def\een{\end{enumerate}}

\def\E{\text{E}}
\def\V{\text{Var}}

\graphicspath{{../figures/}{../figures/book_figures/Chapter3/}{../figures/book_figures/Chapter7/}}

\title[4. Cross-validation]{Business Analytics}
\author{4. Cross-validation}
\date{17 August 2015}

\DefineShortVerb{\"}
\def\FancyVerbFormatCom{\color[rgb]{0.6,0,1}\relax}


\begin{document}

\begin{frame}[plain]{}
\maketitle
\begin{textblock}{11}(0.5,1.19){\color{white}\large
\textbf{ETC3250}}
\end{textblock}
\end{frame}


\section{Choosing regression variables}

\begin{frame}{Choosing regression models}
\biz
\item When there are many predictors, how should we choose which ones to use?

\item How do we choose the df for a spline?

\item We need a way of comparing two competing models

\item If there are a limited number of predictors, we can study all possible models.

\item Otherwise we need a search strategy to explore some potential models.
\eiz

\end{frame}


\begin{frame}{Choosing regression variables}


\structure{What not to do!}
\biz
\item Plot $Y$ against a particular predictor ($X_j$)
       and if it shows no noticeable relationship,  drop it.

%\item Look  at  correlations among the predictors
%       (all of the potential candidates)  and every time a large
%       correlation is encountered,  remove one of  the  two
%       variables from further consideration.

\item Do a multiple linear regression on all the predictors
      and
      disregard all variables whose  $p$ values are greater than 0.05.

\item Maximize $R^2$

\item Minimize SSE.
\eiz
\end{frame}%

\begin{frame}{Comparing regression models}
\structure{Sum of squared errors}
\[\text{SSE} = \sum_{i=1}^n e_i^2 \]
Minimizing SSE will always choose the model with the most predictors.

\pause
\structure{Estimated residual variance}
\[
\hat{\sigma}^2 = \frac{\text{SSE}}{n-k-1}
\]
where $k=$ no.\ predictors.\pause

Minimizing $\hat{\sigma}^2$ works quite well for choosing predictors (but better methods to follow).

\end{frame}


\begin{frame}{Comparing regression models}

However \dots
\begin{itemize}
\item $R^2$  does not allow for ``degrees of freedom''.

\item Adding \textit{any} variable tends to increase the value of $R^2$, even if that variable is
\rlap{irrelevant.}
\end{itemize}\pause

To overcome   this problem, we can use \rlap{\emph{adjusted $R^2$}:}
\begin{block}{}
\[
\bar{R}^2 = 1-(1-R^2)\frac{n-1}{n-k-1}
\]
\end{block}
\pause

\centerline{\textcolor[rgb]{0.8,0.00,0.00}{\textbf{Maximizing $\bar{R}^2$ is equivalent to minimizing $\hat\sigma^2$.}}}

%$\bar{R}^2$    is  referred  to  as
%``adjusted $R^2$'' or ``$R$-bar-squared,''    or sometimes as  ``$R^2$,
%corrected for degrees of freedom.''
\end{frame}


\begin{frame}{Validation sets}
\placefig{0.5}{1.5}{width=12cm,trim=0 0 6 24,clip=true}{cv2}
\begin{textblock}{12}(1.3,1.)
\textcolor{blue}{Training data}\hspace*{4cm}\textcolor{red}{Validation data}
\end{textblock}

\only<2>{\begin{textblock}{12}(0.5,2.9)
\structure{Leave one-out cross-validation (LOOCV)}
\end{textblock}}
\only<2>{\placefig{0.5}{3.5}{width=12cm,trim=0 0 6 0,clip=true}{cvloout}}
\end{frame}

\begin{frame}{LOO Cross-validation}

Leave-one-out cross-validation (LOOCV) for regression can be carried out using the following steps.
\ben
\item Remove observation $i$ from the data set, and fit the model using the remaining data. Then compute the error ($e_i^*=y_i-\hat{y}_i$) for the omitted observation.
\item Repeat step 1 for $i=1,\dots,n$.
\item Compute the MSE from $\{e_1^*,\dots,e_n^*\}$. We shall call this the CV.
\een\pause
The best model is the one with the smallest value of CV.
\end{frame}


\begin{frame}{LOOCV vs validation sets}
\begin{itemize}
\item CV has less bias
\begin{itemize}
\item We repeatedly fit the statistical learning method using training data
that contains $n-1$ obs., i.e. almost all the data set is used
\end{itemize}
\item LOOCV produces a less variable MSE
\begin{itemize}
\item The validation approach produces different MSE when applied
repeatedly due to randomness in the splitting process, while
performing LOOCV multiple times will always yield the same
results, because we split based on 1 obs. each time
\end{itemize}
\item LOOCV is (usually) computationally intensive
\begin{itemize}
\item We fit each model $n$ times! 
\end{itemize}
\end{itemize}
\end{frame}

\begin{frame}{LOOCV for linear models}


\begin{block}{Fitted values}\vspace*{-0.2cm}
\[
\hat{\bm{Y}} =
\text{E}(\bm{Y}|\bm{X}) =
\bm{X}\hat{\bm{\beta}} = \bm{X}(\bm{X}'\bm{X})^{-1}\bm{X}'\bm{Y} = \bm{H}\bm{Y}
\]
\end{block}
where $\bm{H} = \bm{X}(\bm{X}'\bm{X})^{-1}\bm{X}'$ is the ``hat matrix''.\pause

\structure{Leave-one-out residuals}

Let $h_1,\dots,h_n$ be the diagonal values of $\bm{H}$, then the cross-validation statistic is
\begin{block}{}
\centerline{$\displaystyle
\text{CV} = \frac1n\sum_{i=1}^n[e_i/(1-h_i)]^2,$}
\end{block}
where $e_i$ is the residual obtained from fitting the model to all $n$ observations.


\end{frame}

\begin{frame}{Akaike's Information Criterion}

\begin{block}{}
\[
\text{AIC} = -2\log(L) + 2(k+1)
\]
\end{block}
where $L$ is the likelihood and $k$ is the number of predictors in the model.\pause

\biz
\item This is a \emph{penalized likelihood} approach.

\item \emph{Minimizing} the AIC gives the best model for prediction.

\item AIC penalizes terms more heavily than $\bar{R}^2$.

\item Minimizing the AIC is asymptotically equivalent to minimizing MSE via leave-one-out cross-validation.
\eiz

\end{frame}

\begin{frame}{Corrected AIC}

For small values of $n$, the AIC tends to select too many predictors, and so a bias-corrected version of the AIC has been developed.
\begin{block}{}
\[
\text{AIC}_{\text{C}} = \text{AIC} + \frac{2(k+2)(k+3)}{n-k-1}
\]
\end{block}
As with the AIC, the AIC$_{\text{C}}$ should be minimized.
\end{frame}

\begin{frame}{\normalsize Schwartz Bayesian Information Criterion}

\begin{block}{}
\[
\text{BIC} = -2\log(L) + (k+1)\log(n)
\]
\end{block}
where $L$ is the likelihood and $k$ is the number of predictors in the model.\pause

\biz
\item \emph{Minimizing} the BIC gives the best model for prediction.

\item BIC penalizes terms more heavily than \rlap{AIC}

\item Also called SBIC and SC.

\item Minimizing BIC is asymptotically equivalent to leave-$v$-out cross-validation when $v = n[1-1/(log(n)-1)]$.

\eiz

\end{frame}


\begin{frame}{Choosing regression variables}

\structure{Best subsets regression}

\biz
\item
Fit all possible regression models using one or more of the predictors.

\item
Choose the best model based on CV (or an asymptotic equivalent: AIC, AICc).
\eiz

\vfill\pause

\structure{Warning!}
\begin{itemize}
\item If there are a large number of predictors, this is not possible.

For example, 44 predictors leads to 18 trillion possible models!
\end{itemize}

\end{frame}



% \begin{frame}{Choosing regression variables}\footnotesize\vspace*{-0.2cm}
% {\tabcolsep=0.1cm
% \hspace*{-0.3cm}\begin{tabular}{ccccccccc}
%   \hline
%  Savings & Income & Address & Employed & CV & AIC & AICc & BIC & $\bar{R}^2$ \\
%   \hline
%   X & X & X & X & 104.7 & 2325.8 & 2325.9 & 2351.1 & 0.4658 \\
%   X & X & X &   & 106.5 & 2334.1 & 2334.2 & 2355.1 & 0.4558 \\
%   X &   & X & X & 107.7 & 2339.8 & 2339.9 & 2360.9 & 0.4495 \\
%   X &   & X &   & 109.7 & 2349.3 & 2349.3 & 2366.1 & 0.4379 \\
%   X & X &   & X & 112.2 & 2360.4 & 2360.6 & 2381.5 & 0.4263 \\
%   X &   &   & X & 115.1 & 2373.4 & 2373.5 & 2390.3 & 0.4101 \\
%   X & X &   &   & 116.1 & 2377.7 & 2377.8 & 2394.6 & 0.4050 \\
%   X &   &   &   & 119.5 & 2392.1 & 2392.2 & 2404.8 & 0.3864 \\
%     & X & X & X & 164.2 & 2551.6 & 2551.7 & 2572.7 & 0.1592 \\
%     & X & X &   & 164.9 & 2553.8 & 2553.9 & 2570.7 & 0.1538 \\
%     & X &   & X & 176.1 & 2586.7 & 2586.8 & 2603.6 & 0.0963 \\
%     &   & X & X & 177.5 & 2591.4 & 2591.5 & 2608.3 & 0.0877 \\
%     &   & X &   & 178.6 & 2594.6 & 2594.6 & 2607.2 & 0.0801 \\
%     & X &   &   & 179.1 & 2595.3 & 2595.3 & 2607.9 & 0.0788 \\
%     &   &   & X & 190.0 & 2625.3 & 2625.4 & 2638.0 & 0.0217 \\
%     &   &   &   & 193.8 & 2635.3 & 2635.3 & 2643.7 & 0.0000 \\
%    \hline
% \end{tabular}}

% \end{frame}


% \begin{frame}{What to use?}

% \structure{Choice: BIC, CV, AICc, AIC, $\bar{R}^2$.}
% \pause
% \biz
% \item BIC tends to choose models too small for prediction.

% \item As $n\rightarrow\infty$, BIC selects \emph{true} model if there is one.

% \item $\bar{R}^2$ tends to select models too large.

% \item AIC also slightly biased towards larger models (but only when $n$ is small).

% \item Empirical studies AIC is better than BIC for prediction accuracy.
% \eiz
% \end{frame}

\begin{frame}[squeeze,shrink=0.99]{Choosing regression variables}
\structure{Backwards stepwise regression}

\biz
\item Start with a model containing all variables.

\item Try subtracting one variable at a time. Keep the model if it
has lower \rlap{CV, AICc or AIC.}

\item Iterate until no further improvement.
\eiz

\pause%\small

\structure{\textcolor[rgb]{0.80,0.00,0.00}{Notes:}}
\biz\itemsep=0cm\parskip=0cm
%\item Putting all the predictors in can lead to a significant
%\textit{overall effect} but with no way of determining the individual effects
%of each variable.
%
%\item Using stepwise regression is useful for choosing only the most
%significant variables in the regression model.

\item Stepwise regression is not guaranteed to lead to the best possible
model.

\item If you are trying several different models, use the CV, AICc or AIC %adjusted $R^2$
value to select between them.
\eiz

\end{frame}

\section{Cross-validation}


\begin{frame}{Cross-validation}

\begin{itemize}
\item The computational trick for computing LOOCV for linear models doesn't work in other contexts.

\item In general, LOOCV is too computationally intensive. So we use $k$-fold CV instead (where $k=5$ and $k=10$ are common choices).
\end{itemize}

\end{frame}


\begin{frame}{Cross-validation}

\placefig{0.5}{1.5}{width=12cm,trim=0 0 6 24,clip=true}{cv2}
\begin{textblock}{12}(1.3,1.)
\textcolor{blue}{Training data}\hspace*{4cm}\textcolor{red}{Validation data}
\end{textblock}

\only<2>{\begin{textblock}{12}(0.5,2.9)
\structure{Leave one-out cross-validation (LOOCV)}
\end{textblock}}
\only<2>{\placefig{0.5}{3.5}{width=12cm,trim=0 0 6 0,clip=true}{cvloout}}

\only<3>{\begin{textblock}{12}(0.5,2.9)
\structure{5-fold cross-validation}
\end{textblock}}
\only<3>{\placefig{0.5}{3.5}{width=12cm, trim=0 0 6 0,clip=true}{cv2}}
\end{frame}

\begin{frame}{$k$-fold Cross-validation}

\begin{itemize}
\item Divide the data set into $k$ different parts.
\item Remove one part, fit the model on the remaining
$k-1$ parts, and compute the MSE on the omitted part.
\item Repeat $k$ times taking out a different part each time
\end{itemize}\pause

By averaging the $k$ MSEs we get an estimated
validation (test) error rate for new observations.

\begin{block}{}
\[
\text{CV}_{(k)} = \frac{1}{k}\sum_{i=1}^k \text{MSE}_i
\]
\end{block}\pause

LOOCV is a special case where $k=n$.
\end{frame}

\begin{frame}{$k$-fold Cross-validation}
\begin{itemize}
\item Each training set is only $(k-1)/k$ as big as the original data set. So the estimates of prediction error will be biased upwards.
\item Bias minimized when $k=n$ (LOOCV).

\item But variance increases with $k$ (as there are overlapping observations in each part).

\item $k=5$ or $k=10$ provide a good compromise for this bias-variance tradeoff.
\end{itemize}
\end{frame}
\end{document}