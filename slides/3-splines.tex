% * Week 3. (Aug 10) Regression for prediction. Ch3 (Rob)
%   - Lecture 5: Review of linear regression, matrix formulation
%   - Lab 3:
%   - Lecture 6: Subset selection, LOOCV 


\documentclass[14pt]{beamer}
\usepackage{pgf,tikz,pgfpages,amsmath,bm,fancyvrb,animate}
\usepackage{graphicx,bera,booktabs}
\usepackage[australian]{babel}
\usepackage[utf8]{inputenc}

\usetheme{Monash}
\def\biz{\begin{itemize}[<+-| alert@+>]}
\def\eiz{\end{itemize}}
\def\ben{\begin{enumerate}[<+-| alert@+>]}
\def\een{\end{enumerate}}

\def\E{\text{E}}
\def\V{\text{Var}}

\graphicspath{{../figures/}{../figures/book_figures/Chapter3/}{../figures/book_figures/Chapter7/}}

\title[3. Nonparametric regression]{Business Analytics}
\author{3. Nonparametric regression}
\date{13 August 2015}

\DefineShortVerb{\"}
\def\FancyVerbFormatCom{\color[rgb]{0.6,0,1}\relax}


\begin{document}

\begin{frame}[plain]{}
\maketitle
\begin{textblock}{11}(0.5,1.19){\color{white}\large
\textbf{ETC3250}}
\end{textblock}
\end{frame}

\section{Moving beyond linearity}

\begin{frame}{Moving beyond linearity}
The truth is never linear!\pause

Or almost never!\pause

But often the linearity assumption is good enough.\pause

When it's not \dots
\begin{itemize}
\item polynomials
\item step functions,
\item splines,
\item local regression, and
\item generalized additive models
\end{itemize}
offer a lot of flexibility, without losing the ease and
interpretability of linear models.

\end{frame}
\begin{frame}{Nonlinear choices}
\begin{enumerate}
\item Polynomials (beware)
\item Truncated power basis splines
\item Natural splines
\item B-splines
\item Smoothing splines
\item Radial basis functions
\item Kernel regression
\item Local regression
\item kNN
\end{enumerate}
\end{frame}

\section{Splines}

\begin{frame}{Splines}
\only<1>{\fullheight{draftspline}}
\only<2>{\fullwidth{spline}}
\only<3>{\fullwidth{spline2}}
\end{frame}
 
\begin{frame}{Splines}\parskip=1.6ex

Knots: $\kappa_1,\dots,\kappa_K$.

\pause

A spline is a continuous function $f(x)$ consisting of polynomials between each
consecutive pair of `knots' $x=\kappa_j$ and $x=\kappa_{j+1}$.

\pause

\begin{itemize}
\item Parameters constrained so that $f(x)$ is continuous.

\item Further constraints imposed to give continuous derivatives.
\end{itemize}
\end{frame}


\begin{frame}{Splines}
\fullheight{7-3}

\end{frame}



\begin{frame}{Truncated power basis}
\biz
%\item
%Let $\kappa_1<\kappa_2<\cdots<\kappa_k$ be knots in interval $(a,b)$.

\item Predictors: $x$, \dots, $x^p$, $(x-\kappa_{1})_+^p$, \dots, $(x-\kappa_{K})_+^p$

\item Then the regression is piecewise order-$p$ polynomials.
\item $p-1$ continuous derivatives.
\item Usually choose $p=1$ or $p=3$.

\item $p+K+1$ degrees of freedom
\eiz
\end{frame}


\begin{frame}[fragile]{Natural splines}

\begin{itemize}
\item 
Splines based on truncated power bases have high variance at the outer range of the predictors.
\item Natural splines are similar, but have additional \alert{boundary constraints}: the function is linear at the boundaries. This reduces the variance.

\item Degrees of freedom $\verb|df|=K$.

\item Create predictors using \verb|ns| function in R (automatically chooses knots given \verb|df|).


\end{itemize}

\end{frame}

\begin{frame}{Natural splines}
\only<1>{\fullheight{7-4}}

\only<2>{\fullheight{7-7}}
\end{frame}

\begin{frame}[fragile]{Knot placement}

\begin{itemize}
\item Strategy 1: specify \verb|df| (equivalently $K$) and let \verb|ns| place them at appropriate quantiles of the observed $X$.

\item Strategy 2: choose $K$ and their locations.
\end{itemize}

\end{frame}

\begin{frame}{Splines}
\fullwidth{pulpspline}
\only<2>{\fullwidth{pulpspline2}}
\only<3>{\fullwidth{pulpspline3}}
\only<4>{\fullwidth{pulpspline4}}
\only<5>{\fullwidth{pulpspline5}}
\only<6>{\fullwidth{pulpspline6}}
\only<7>{\fullwidth{pulpspline7}}
\only<8>{\fullwidth{pulpspline8}}
\only<9>{\fullwidth{pulpspline9}}
\only<10>{\fullwidth{pulpspline10}}
\only<11>{\fullwidth{pulpspline11}}
\only<12>{\fullwidth{pulpspline12}}
\only<13>{\fullwidth{pulpspline13}}
\only<14>{\fullwidth{pulpspline14}}
\only<15>{\fullwidth{pulpspline15}}
\only<16>{\fullwidth{pulpspline16}}
\only<17>{\fullwidth{pulpspline17}}
\only<18>{\fullwidth{pulpspline18}}
\only<19>{\fullwidth{pulpspline19}}
\only<20>{\fullwidth{pulpspline20}}
\only<21>{\fullwidth{pulpspline21}}
\only<22>{\fullwidth{pulpspline22}}
\only<23>{\fullwidth{pulpspline23}}
\only<24>{\fullwidth{pulpspline24}}
\only<25>{\fullwidth{pulpspline25}}
\end{frame}

\begin{frame}{Basis functions}
\only<1>{\fullwidth{polybasis}}
\only<2>{\fullwidth{truncpolybasis}}
\only<3>{\fullwidth{nsbasis}}
\only<4>{\fullwidth{bsbasis}}
\end{frame}


\section{Generalized Additive Models}


\begin{frame}{The curse of dimensionality}

Why can't we fit models of the form
$$y = f(x_1,x_2,\dots,x_p) + e?$$\pause

\begin{itemize}
\item Data is very sparse in high-dimensional space.
\item Model assumes $p$-way interactions which are almost impossible to estimate.
\end{itemize}
\end{frame}

\begin{frame}{Generalized Additive Models}
Allows for flexible nonlinearities in several variables, but retains
the additive structure of linear models.

\begin{block}{}\vspace*{-0.2cm}
$$y_i = \beta_0 + f_1(x_{i,1}) + f_2(x_{i,2}) + \dots + f_p(x_{p,1}) + e_i$$
\end{block}\pause

\begin{itemize}
\item Each $f_i$ is a smooth univariate function.
\end{itemize}
\end{frame}

\begin{frame}[fragile]{Generalized Additive Models}

\begin{itemize}
\item Can fit a GAM simply using, e.g. natural splines:
\begin{footnotesize}
\begin{BVerbatim}
lm(wage ~ ns(year,df=5) + ns(age,df=5) + education)
\end{BVerbatim}
\end{footnotesize}
\item Coefficients not that interesting; fitted functions are. 
\item Use \verb|plot.gam| from \verb|gam| package.
\item  Can mix terms --- some linear, some nonlinear --- and use
\verb|anova()| to compare models.
\item GAMs are additive, although low-order interactions can be
included in a natural way using, e.g. bivariate smoothers or
interactions of the form ns(age,df=5):ns(year,df=5).
\end{itemize}
\end{frame}


\begin{frame}[fragile]{Interactions and additivity}

\begin{itemize}
\item Additive models assume no interactions. 
\item Add bivariate smooths for two-way interactions.
\item Graphically check for interactions using faceting.
\end{itemize}
\pause

\begin{block}{}\small
\begin{BVerbatim}
qplot(age, wage, data = Wage) + facet_wrap(~ year)
\end{BVerbatim}
\end{block}
\end{frame}
\end{document}